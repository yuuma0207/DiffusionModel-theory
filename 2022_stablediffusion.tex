%ctrl+alt+m:open Math Preview
\documentclass[a4paper,11pt,uplatex]{jsarticle}%titlepage
%:/usr/local/texlive/texmf-local/tex/latex/report/report.sty
\usepackage{myreport}
\title{生成モデルの数理情報学}
\author{20B01392 松本侑真}
\date{\today}
\begin{document}
\maketitle
\begin{abstract}
近年、機械学習を応用した生成系AIがブームである。
生成系AIを使うことで、プロンプトに入力したテキストデータから画像データなどを出力することができる。
有名な生成系AIであるStable Diffusionでは、無料でプロンプトに入力した文字(例:かわいい犬)から、かわいい犬の画像を出力することができる。
このような生成系AIを実装するにあたり、どのような理論的背景が存在するかを説明する。
\end{abstract}
\tableofcontents
\newpage

\section{生成モデルとボルツマンマシン}

\section{隠れ変数を導入してリッチなモデルへ}

\section{制限ボルツマンマシン}

\section{マルコフ連鎖モンテカルロ法(MCMC)}

\section{交換モンテカルロ法}

\section{MCMCを用いないボルツマンマシン}

\section{変分オートエンコーダー}

\section{階層変分オートエンコーダ}

\end{document}